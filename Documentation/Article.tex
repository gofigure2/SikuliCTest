%
% Complete documentation on the extended LaTeX markup used for Insight
% documentation is available in ``Documenting Insight'', which is part
% of the standard documentation for Insight.  It may be found online
% at:
%
%     http://www.itk.org/

\documentclass{InsightArticle}

\usepackage[dvips]{graphicx}


% Personal settings for code snippets visualization
\input{settings/TexCodeSet}


%%%%%%%%%%%%%%%%%%%%%%%%%%%%%%%%%%%%%%%%%%%%%%%%%%%%%%%%%%%%%%%%%%
%
%  hyperref should be the last package to be loaded.
%
%%%%%%%%%%%%%%%%%%%%%%%%%%%%%%%%%%%%%%%%%%%%%%%%%%%%%%%%%%%%%%%%%%
\usepackage[dvips,
bookmarks,
bookmarksopen,
backref,
colorlinks,linkcolor={blue},citecolor={blue},urlcolor={blue},
]{hyperref}


%  This is a template for Papers to the Insight Journal. 
%  It is comparable to a technical report format.

% The title should be descriptive enough for people to be able to find
% the relevant document.
\title{CTest Integration of Sikuli Automated GUI Testing}

% 
% NOTE: This is the last number of the "handle" URL that 
% The Insight Journal assigns to your paper as part of the
% submission process. Please replace the number "1338" with
% the actual handle number that you get assigned.
%
\newcommand{\IJhandlerIDnumber}{3196}

% Increment the release number whenever significant changes are made.
% The author and/or editor can define 'significant' however they like.
\release{0.00}

% At minimum, give your name and an email address.  You can include a
% snail-mail address if you like.
\author{Evan Schwab, Lydie Souhait, Nicolas Rannou, Kishore Mosaliganti\\
Arnaud Gelas, Sean Megason}
\authoraddress{Harvard Medical School, Megason lab}

\begin{document}

% Add hyperlink to the web location and license of the paper.
\IJhandlefooter{\IJhandlerIDnumber}


\ifpdf
\else
   %
   % Commands for including Graphics when using latex
   %
   \DeclareGraphicsExtensions{.eps,.jpg,.gif,.tiff,.bmp,.png}
   \DeclareGraphicsRule{.jpg}{eps}{.jpg.bb}{`convert #1 eps:-}
   \DeclareGraphicsRule{.gif}{eps}{.gif.bb}{`convert #1 eps:-}
   \DeclareGraphicsRule{.tiff}{eps}{.tiff.bb}{`convert #1 eps:-}
   \DeclareGraphicsRule{.bmp}{eps}{.bmp.bb}{`convert #1 eps:-}
   \DeclareGraphicsRule{.png}{eps}{.png.bb}{`convert #1 eps:-}
\fi

\maketitle

\ifhtml
\chapter*{Front Matter\label{front}}
\fi

\begin{abstract}
\noindent
\end{abstract}

\IJhandlenote{\IJhandlerIDnumber}

\tableofcontents
% ------------------------------------------------------------------------
\section{Introduction}

% ------------------------------------------------------------------------
\section{Sikuli}

% Feel free to change section titles
\subsection{What is sikuli?}
Sikuli is a programming language that uses screen shot images as variables
and objects in order to automate graphical user interface functions. The
Sikuli program is comprised of an IDE where the user can write scripts that 
include screen shot thumbnails so they can visually track the functions of
their code. The Sikuli Script is built on a Jython (Python for Java platform)
library which uses Python syntax in addition to a number of special Sikuli
functions for aquiring and handling screenshot images and performing mouse,
keyboard actions, the most useful of which are conveniently found as buttons
on the Sikuli IDE.     

One of Sikuli's primary uses is the automation of GUI testing for software
developers.


\subsection{Simple of example} % not related to gofigure...

\begin{figure}[htp]
 \centering
 
 \caption{here is my caption}
 \label{fig:SimpleExample}
\end{figure}

% ------------------------------------------------------------------------
\section{Integration with CTest}
% I'll write it this part

% ------------------------------------------------------------------------
\section{Example}

% ------------------------------------------------------------------------
\section{Conclusion}

\clearpage

\bibliographystyle{plain}
\bibliography{InsightJournal,AntoBib}


\end{document}

